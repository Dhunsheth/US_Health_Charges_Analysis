% Options for packages loaded elsewhere
\PassOptionsToPackage{unicode}{hyperref}
\PassOptionsToPackage{hyphens}{url}
%
\documentclass[
]{article}
\usepackage{amsmath,amssymb}
\usepackage{iftex}
\ifPDFTeX
  \usepackage[T1]{fontenc}
  \usepackage[utf8]{inputenc}
  \usepackage{textcomp} % provide euro and other symbols
\else % if luatex or xetex
  \usepackage{unicode-math} % this also loads fontspec
  \defaultfontfeatures{Scale=MatchLowercase}
  \defaultfontfeatures[\rmfamily]{Ligatures=TeX,Scale=1}
\fi
\usepackage{lmodern}
\ifPDFTeX\else
  % xetex/luatex font selection
\fi
% Use upquote if available, for straight quotes in verbatim environments
\IfFileExists{upquote.sty}{\usepackage{upquote}}{}
\IfFileExists{microtype.sty}{% use microtype if available
  \usepackage[]{microtype}
  \UseMicrotypeSet[protrusion]{basicmath} % disable protrusion for tt fonts
}{}
\makeatletter
\@ifundefined{KOMAClassName}{% if non-KOMA class
  \IfFileExists{parskip.sty}{%
    \usepackage{parskip}
  }{% else
    \setlength{\parindent}{0pt}
    \setlength{\parskip}{6pt plus 2pt minus 1pt}}
}{% if KOMA class
  \KOMAoptions{parskip=half}}
\makeatother
\usepackage{xcolor}
\usepackage[margin=1in]{geometry}
\usepackage{color}
\usepackage{fancyvrb}
\newcommand{\VerbBar}{|}
\newcommand{\VERB}{\Verb[commandchars=\\\{\}]}
\DefineVerbatimEnvironment{Highlighting}{Verbatim}{commandchars=\\\{\}}
% Add ',fontsize=\small' for more characters per line
\usepackage{framed}
\definecolor{shadecolor}{RGB}{248,248,248}
\newenvironment{Shaded}{\begin{snugshade}}{\end{snugshade}}
\newcommand{\AlertTok}[1]{\textcolor[rgb]{0.94,0.16,0.16}{#1}}
\newcommand{\AnnotationTok}[1]{\textcolor[rgb]{0.56,0.35,0.01}{\textbf{\textit{#1}}}}
\newcommand{\AttributeTok}[1]{\textcolor[rgb]{0.13,0.29,0.53}{#1}}
\newcommand{\BaseNTok}[1]{\textcolor[rgb]{0.00,0.00,0.81}{#1}}
\newcommand{\BuiltInTok}[1]{#1}
\newcommand{\CharTok}[1]{\textcolor[rgb]{0.31,0.60,0.02}{#1}}
\newcommand{\CommentTok}[1]{\textcolor[rgb]{0.56,0.35,0.01}{\textit{#1}}}
\newcommand{\CommentVarTok}[1]{\textcolor[rgb]{0.56,0.35,0.01}{\textbf{\textit{#1}}}}
\newcommand{\ConstantTok}[1]{\textcolor[rgb]{0.56,0.35,0.01}{#1}}
\newcommand{\ControlFlowTok}[1]{\textcolor[rgb]{0.13,0.29,0.53}{\textbf{#1}}}
\newcommand{\DataTypeTok}[1]{\textcolor[rgb]{0.13,0.29,0.53}{#1}}
\newcommand{\DecValTok}[1]{\textcolor[rgb]{0.00,0.00,0.81}{#1}}
\newcommand{\DocumentationTok}[1]{\textcolor[rgb]{0.56,0.35,0.01}{\textbf{\textit{#1}}}}
\newcommand{\ErrorTok}[1]{\textcolor[rgb]{0.64,0.00,0.00}{\textbf{#1}}}
\newcommand{\ExtensionTok}[1]{#1}
\newcommand{\FloatTok}[1]{\textcolor[rgb]{0.00,0.00,0.81}{#1}}
\newcommand{\FunctionTok}[1]{\textcolor[rgb]{0.13,0.29,0.53}{\textbf{#1}}}
\newcommand{\ImportTok}[1]{#1}
\newcommand{\InformationTok}[1]{\textcolor[rgb]{0.56,0.35,0.01}{\textbf{\textit{#1}}}}
\newcommand{\KeywordTok}[1]{\textcolor[rgb]{0.13,0.29,0.53}{\textbf{#1}}}
\newcommand{\NormalTok}[1]{#1}
\newcommand{\OperatorTok}[1]{\textcolor[rgb]{0.81,0.36,0.00}{\textbf{#1}}}
\newcommand{\OtherTok}[1]{\textcolor[rgb]{0.56,0.35,0.01}{#1}}
\newcommand{\PreprocessorTok}[1]{\textcolor[rgb]{0.56,0.35,0.01}{\textit{#1}}}
\newcommand{\RegionMarkerTok}[1]{#1}
\newcommand{\SpecialCharTok}[1]{\textcolor[rgb]{0.81,0.36,0.00}{\textbf{#1}}}
\newcommand{\SpecialStringTok}[1]{\textcolor[rgb]{0.31,0.60,0.02}{#1}}
\newcommand{\StringTok}[1]{\textcolor[rgb]{0.31,0.60,0.02}{#1}}
\newcommand{\VariableTok}[1]{\textcolor[rgb]{0.00,0.00,0.00}{#1}}
\newcommand{\VerbatimStringTok}[1]{\textcolor[rgb]{0.31,0.60,0.02}{#1}}
\newcommand{\WarningTok}[1]{\textcolor[rgb]{0.56,0.35,0.01}{\textbf{\textit{#1}}}}
\usepackage{longtable,booktabs,array}
\usepackage{calc} % for calculating minipage widths
% Correct order of tables after \paragraph or \subparagraph
\usepackage{etoolbox}
\makeatletter
\patchcmd\longtable{\par}{\if@noskipsec\mbox{}\fi\par}{}{}
\makeatother
% Allow footnotes in longtable head/foot
\IfFileExists{footnotehyper.sty}{\usepackage{footnotehyper}}{\usepackage{footnote}}
\makesavenoteenv{longtable}
\usepackage{graphicx}
\makeatletter
\def\maxwidth{\ifdim\Gin@nat@width>\linewidth\linewidth\else\Gin@nat@width\fi}
\def\maxheight{\ifdim\Gin@nat@height>\textheight\textheight\else\Gin@nat@height\fi}
\makeatother
% Scale images if necessary, so that they will not overflow the page
% margins by default, and it is still possible to overwrite the defaults
% using explicit options in \includegraphics[width, height, ...]{}
\setkeys{Gin}{width=\maxwidth,height=\maxheight,keepaspectratio}
% Set default figure placement to htbp
\makeatletter
\def\fps@figure{htbp}
\makeatother
\setlength{\emergencystretch}{3em} % prevent overfull lines
\providecommand{\tightlist}{%
  \setlength{\itemsep}{0pt}\setlength{\parskip}{0pt}}
\setcounter{secnumdepth}{-\maxdimen} % remove section numbering
\ifLuaTeX
  \usepackage{selnolig}  % disable illegal ligatures
\fi
\IfFileExists{bookmark.sty}{\usepackage{bookmark}}{\usepackage{hyperref}}
\IfFileExists{xurl.sty}{\usepackage{xurl}}{} % add URL line breaks if available
\urlstyle{same}
\hypersetup{
  pdftitle={ US Health Insurance Charges },
  pdfauthor={ Dhun Sheth },
  hidelinks,
  pdfcreator={LaTeX via pandoc}}

\title{\\
US Health Insurance Charges\\}
\author{\strut \\
Dhun Sheth\\}
\date{\strut \\
2023-12-19\\}

\begin{document}
\maketitle

\hypertarget{introduction}{%
\subsection{Introduction}\label{introduction}}

Below is an attempt to model individuals health insurance charges in the
United States based on some demographic information (BMI, Region,
Smoker, Children, Sex, and Age).

\hypertarget{setup}{%
\subsection{Setup}\label{setup}}

Splitting randomized data into 2/3 train and 1/3 test set.

\begin{Shaded}
\begin{Highlighting}[]
\FunctionTok{set.seed}\NormalTok{(}\DecValTok{4}\NormalTok{)}
\NormalTok{insurance\_data }\OtherTok{\textless{}{-}} \FunctionTok{read.csv}\NormalTok{(}\StringTok{"insurance.csv"}\NormalTok{, }\AttributeTok{stringsAsFactors=}\ConstantTok{TRUE}\NormalTok{)}
\FunctionTok{print}\NormalTok{(}\FunctionTok{summary}\NormalTok{(insurance\_data))}
\end{Highlighting}
\end{Shaded}

\begin{verbatim}
##       age            sex           bmi           children     smoker    
##  Min.   :18.00   female:662   Min.   :15.96   Min.   :0.000   no :1064  
##  1st Qu.:27.00   male  :676   1st Qu.:26.30   1st Qu.:0.000   yes: 274  
##  Median :39.00                Median :30.40   Median :1.000             
##  Mean   :39.21                Mean   :30.66   Mean   :1.095             
##  3rd Qu.:51.00                3rd Qu.:34.69   3rd Qu.:2.000             
##  Max.   :64.00                Max.   :53.13   Max.   :5.000             
##        region       charges     
##  northeast:324   Min.   : 1122  
##  northwest:325   1st Qu.: 4740  
##  southeast:364   Median : 9382  
##  southwest:325   Mean   :13270  
##                  3rd Qu.:16640  
##                  Max.   :63770
\end{verbatim}

\begin{Shaded}
\begin{Highlighting}[]
\NormalTok{data\_randomized }\OtherTok{\textless{}{-}} \FunctionTok{sample}\NormalTok{(insurance\_data)}

\NormalTok{d\_train }\OtherTok{\textless{}{-}}\NormalTok{ data\_randomized[}\DecValTok{1}\SpecialCharTok{:}\NormalTok{(}\DecValTok{2}\SpecialCharTok{*}\FunctionTok{length}\NormalTok{(data\_randomized}\SpecialCharTok{$}\NormalTok{charges)}\SpecialCharTok{/}\DecValTok{3}\NormalTok{),]}
\NormalTok{d\_test }\OtherTok{\textless{}{-}}\NormalTok{ data\_randomized[(}\DecValTok{2}\SpecialCharTok{*}\FunctionTok{length}\NormalTok{(data\_randomized}\SpecialCharTok{$}\NormalTok{charges)}\SpecialCharTok{/}\DecValTok{3}\SpecialCharTok{+}\DecValTok{1}\NormalTok{)}\SpecialCharTok{:}\NormalTok{(}\FunctionTok{length}\NormalTok{(data\_randomized}\SpecialCharTok{$}\NormalTok{charges)),]}
\end{Highlighting}
\end{Shaded}

\hypertarget{linear-model}{%
\subsubsection{Linear Model}\label{linear-model}}

\begin{Shaded}
\begin{Highlighting}[]
\NormalTok{lm\_insu\_reg }\OtherTok{\textless{}{-}} \FunctionTok{lm}\NormalTok{(charges}\SpecialCharTok{\textasciitilde{}}\NormalTok{., }\AttributeTok{data=}\NormalTok{d\_train)}
\FunctionTok{summary}\NormalTok{(lm\_insu\_reg)}
\end{Highlighting}
\end{Shaded}

\begin{verbatim}
## 
## Call:
## lm(formula = charges ~ ., data = d_train)
## 
## Residuals:
##      Min       1Q   Median       3Q      Max 
## -11216.1  -2806.9   -861.2   1307.5  24938.4 
## 
## Coefficients:
##                  Estimate Std. Error t value Pr(>|t|)    
## (Intercept)     -11968.80    1208.90  -9.901   <2e-16 ***
## bmi                344.80      34.54   9.984   <2e-16 ***
## regionnorthwest   -326.56     575.88  -0.567   0.5708    
## regionsoutheast  -1083.18     564.02  -1.920   0.0551 .  
## regionsouthwest   -919.89     571.83  -1.609   0.1080    
## smokeryes        24110.11     501.28  48.097   <2e-16 ***
## children           400.41     167.66   2.388   0.0171 *  
## sexmale           -595.33     399.69  -1.489   0.1367    
## age                256.64      14.11  18.183   <2e-16 ***
## ---
## Signif. codes:  0 '***' 0.001 '**' 0.01 '*' 0.05 '.' 0.1 ' ' 1
## 
## Residual standard error: 5935 on 883 degrees of freedom
## Multiple R-squared:  0.7614, Adjusted R-squared:  0.7592 
## F-statistic: 352.2 on 8 and 883 DF,  p-value: < 2.2e-16
\end{verbatim}

\begin{Shaded}
\begin{Highlighting}[]
\FunctionTok{par}\NormalTok{(}\AttributeTok{mfrow =} \FunctionTok{c}\NormalTok{(}\DecValTok{2}\NormalTok{, }\DecValTok{2}\NormalTok{))}
\FunctionTok{plot}\NormalTok{(lm\_insu\_reg)}
\end{Highlighting}
\end{Shaded}

\includegraphics{us_health_insurance_charges_project_files/figure-latex/question_3_linear-1.pdf}

\begin{Shaded}
\begin{Highlighting}[]
\FunctionTok{par}\NormalTok{(}\AttributeTok{mfrow =} \FunctionTok{c}\NormalTok{(}\DecValTok{1}\NormalTok{, }\DecValTok{1}\NormalTok{))}

\NormalTok{cvlm }\OtherTok{\textless{}{-}} \FunctionTok{list}\NormalTok{()}
\NormalTok{msecv }\OtherTok{\textless{}{-}} \ConstantTok{NA}
\ControlFlowTok{for}\NormalTok{(i }\ControlFlowTok{in} \DecValTok{1}\SpecialCharTok{:}\FunctionTok{nrow}\NormalTok{(d\_train))\{}
\NormalTok{  cvlm[[i]] }\OtherTok{\textless{}{-}} \FunctionTok{lm}\NormalTok{(charges}\SpecialCharTok{\textasciitilde{}}\NormalTok{., }\AttributeTok{data=}\NormalTok{d\_train[}\SpecialCharTok{{-}}\NormalTok{i,])}
\NormalTok{  msecv[i] }\OtherTok{\textless{}{-}}\NormalTok{ (}\FunctionTok{predict}\NormalTok{(cvlm[[i]], }\AttributeTok{newdata=}\NormalTok{d\_train[i,]) }\SpecialCharTok{{-}}\NormalTok{ d\_train}\SpecialCharTok{$}\NormalTok{charges[[i]])}\SpecialCharTok{\^{}}\DecValTok{2}
\NormalTok{\}}
\NormalTok{lm\_cv\_MSE }\OtherTok{\textless{}{-}} \FunctionTok{mean}\NormalTok{(msecv)}
\FunctionTok{sprintf}\NormalTok{(}\StringTok{"CV MSE estimate for linear model on training data: \%s"}\NormalTok{, }\FunctionTok{round}\NormalTok{(lm\_cv\_MSE, }\AttributeTok{digits=}\DecValTok{3}\NormalTok{))}
\end{Highlighting}
\end{Shaded}

\begin{verbatim}
## [1] "CV MSE estimate for linear model on training data: 35678999.236"
\end{verbatim}

\begin{Shaded}
\begin{Highlighting}[]
\NormalTok{predicted\_vals\_test }\OtherTok{\textless{}{-}} \FunctionTok{predict}\NormalTok{(lm\_insu\_reg, }\AttributeTok{newdata =}\NormalTok{ d\_test)}
\NormalTok{lm\_test\_MSE }\OtherTok{\textless{}{-}} \FunctionTok{mean}\NormalTok{((d\_test}\SpecialCharTok{$}\NormalTok{charges }\SpecialCharTok{{-}}\NormalTok{ predicted\_vals\_test)}\SpecialCharTok{\^{}}\DecValTok{2}\NormalTok{)}
\FunctionTok{sprintf}\NormalTok{(}\StringTok{"MSE estimate for linear model on testing data: \%s"}\NormalTok{, }\FunctionTok{round}\NormalTok{(lm\_test\_MSE, }\AttributeTok{digits=}\DecValTok{3}\NormalTok{))}
\end{Highlighting}
\end{Shaded}

\begin{verbatim}
## [1] "MSE estimate for linear model on testing data: 40020943.282"
\end{verbatim}

Based on the regression plots, the residuals don't seem independent or
normally distributed or have constant variance which violate many of the
assumptions necessary to fit a linear model.

\hypertarget{lasso}{%
\subsubsection{Lasso}\label{lasso}}

\begin{Shaded}
\begin{Highlighting}[]
\NormalTok{x }\OtherTok{\textless{}{-}} \FunctionTok{as.matrix}\NormalTok{(d\_train[, }\SpecialCharTok{{-}}\DecValTok{2}\NormalTok{])}
\NormalTok{y }\OtherTok{\textless{}{-}} \FunctionTok{as.vector}\NormalTok{(d\_train[, }\DecValTok{2}\NormalTok{])}


\NormalTok{lasso\_reg }\OtherTok{\textless{}{-}} \FunctionTok{cv.glmnet}\NormalTok{(x,y, }\AttributeTok{alpha =} \DecValTok{1}\NormalTok{, }\AttributeTok{lamda =}\NormalTok{ lambda\_values)}
\FunctionTok{plot}\NormalTok{(lasso\_reg)}
\end{Highlighting}
\end{Shaded}

\includegraphics{us_health_insurance_charges_project_files/figure-latex/question_3_lasso-1.pdf}

\begin{Shaded}
\begin{Highlighting}[]
\NormalTok{lasso\_cv\_mse }\OtherTok{\textless{}{-}} \FunctionTok{min}\NormalTok{(lasso\_reg}\SpecialCharTok{$}\NormalTok{cvm)}
\FunctionTok{print}\NormalTok{(lasso\_cv\_mse)}
\end{Highlighting}
\end{Shaded}

\begin{verbatim}
## [1] 128107020
\end{verbatim}

\begin{Shaded}
\begin{Highlighting}[]
\NormalTok{best\_lasso\_reg }\OtherTok{\textless{}{-}} \FunctionTok{glmnet}\NormalTok{(x, y, }\AttributeTok{alpha =} \DecValTok{1}\NormalTok{, }\AttributeTok{lambda =}\NormalTok{ lasso\_reg}\SpecialCharTok{$}\NormalTok{lambda.min)}

\NormalTok{predicted\_vals\_test }\OtherTok{\textless{}{-}} \FunctionTok{predict}\NormalTok{(best\_lasso\_reg, }\AttributeTok{newx =} \FunctionTok{as.matrix}\NormalTok{(d\_test[,}\SpecialCharTok{{-}}\DecValTok{2}\NormalTok{]))}
\NormalTok{lasso\_test\_MSE }\OtherTok{\textless{}{-}} \FunctionTok{mean}\NormalTok{((d\_test}\SpecialCharTok{$}\NormalTok{charges }\SpecialCharTok{{-}}\NormalTok{ predicted\_vals\_test)}\SpecialCharTok{\^{}}\DecValTok{2}\NormalTok{)}
\FunctionTok{sprintf}\NormalTok{(}\StringTok{"MSE estimate for Lasso regression on testing data: \%s"}\NormalTok{, }\FunctionTok{round}\NormalTok{(lasso\_test\_MSE, }\AttributeTok{digits=}\DecValTok{3}\NormalTok{))}
\end{Highlighting}
\end{Shaded}

\begin{verbatim}
## [1] "MSE estimate for Lasso regression on testing data: 133521776.277"
\end{verbatim}

\hypertarget{trees}{%
\subsubsection{Trees}\label{trees}}

\begin{Shaded}
\begin{Highlighting}[]
\NormalTok{tree\_insu\_reg }\OtherTok{\textless{}{-}} \FunctionTok{tree}\NormalTok{(charges}\SpecialCharTok{\textasciitilde{}}\NormalTok{., }\AttributeTok{data =}\NormalTok{ d\_train)}

\FunctionTok{set.seed}\NormalTok{(}\DecValTok{51341}\NormalTok{)}
\NormalTok{tree\_insu\_cv }\OtherTok{\textless{}{-}} \FunctionTok{cv.tree}\NormalTok{(tree\_insu\_reg)}
\FunctionTok{plot}\NormalTok{(tree\_insu\_cv, }\AttributeTok{type=}\StringTok{"b"}\NormalTok{) }
\end{Highlighting}
\end{Shaded}

\includegraphics{us_health_insurance_charges_project_files/figure-latex/question_3_trees-1.pdf}

\begin{Shaded}
\begin{Highlighting}[]
\FunctionTok{print}\NormalTok{(}\StringTok{"No pruning needed"}\NormalTok{)}
\end{Highlighting}
\end{Shaded}

\begin{verbatim}
## [1] "No pruning needed"
\end{verbatim}

\begin{Shaded}
\begin{Highlighting}[]
\NormalTok{tree\_cv\_MSE }\OtherTok{\textless{}{-}} \FunctionTok{min}\NormalTok{(tree\_insu\_cv}\SpecialCharTok{$}\NormalTok{dev)}\SpecialCharTok{/}\FunctionTok{length}\NormalTok{(insurance\_data}\SpecialCharTok{$}\NormalTok{charges)}
\FunctionTok{sprintf}\NormalTok{(}\StringTok{"Tree CV MSE estimate based on training data: \%s"}\NormalTok{, }\FunctionTok{round}\NormalTok{(tree\_cv\_MSE,}\AttributeTok{digits=}\DecValTok{3}\NormalTok{)) }\CommentTok{\# Tree CV MSE Estimate}
\end{Highlighting}
\end{Shaded}

\begin{verbatim}
## [1] "Tree CV MSE estimate based on training data: 16522123.162"
\end{verbatim}

\begin{Shaded}
\begin{Highlighting}[]
\FunctionTok{plot}\NormalTok{(tree\_insu\_reg)}
\FunctionTok{text}\NormalTok{(tree\_insu\_reg, }\AttributeTok{pretty=}\DecValTok{0}\NormalTok{)}
\end{Highlighting}
\end{Shaded}

\includegraphics{us_health_insurance_charges_project_files/figure-latex/question_3_trees-2.pdf}

\begin{Shaded}
\begin{Highlighting}[]
\FunctionTok{summary}\NormalTok{(tree\_insu\_reg)}
\end{Highlighting}
\end{Shaded}

\begin{verbatim}
## 
## Regression tree:
## tree(formula = charges ~ ., data = d_train)
## Variables actually used in tree construction:
## [1] "smoker" "age"    "bmi"   
## Number of terminal nodes:  4 
## Residual mean deviance:  24120000 = 2.142e+10 / 888 
## Distribution of residuals:
##    Min. 1st Qu.  Median    Mean 3rd Qu.    Max. 
##   -9381   -3101   -1037       0    1202   22990
\end{verbatim}

\begin{Shaded}
\begin{Highlighting}[]
\NormalTok{predicted\_vals\_test }\OtherTok{\textless{}{-}} \FunctionTok{predict}\NormalTok{(tree\_insu\_reg, }\AttributeTok{newdata =}\NormalTok{ d\_test)}
\NormalTok{tree\_test\_MSE }\OtherTok{\textless{}{-}} \FunctionTok{mean}\NormalTok{((d\_test}\SpecialCharTok{$}\NormalTok{charges }\SpecialCharTok{{-}}\NormalTok{ predicted\_vals\_test)}\SpecialCharTok{\^{}}\DecValTok{2}\NormalTok{)}

\FunctionTok{sprintf}\NormalTok{(}\StringTok{"Tree MSE estimate based on testing data: \%s"}\NormalTok{, }\FunctionTok{round}\NormalTok{(tree\_test\_MSE,}\AttributeTok{digits=}\DecValTok{3}\NormalTok{))}
\end{Highlighting}
\end{Shaded}

\begin{verbatim}
## [1] "Tree MSE estimate based on testing data: 28823125.572"
\end{verbatim}

\hypertarget{random-forest}{%
\subsubsection{Random Forest}\label{random-forest}}

\begin{Shaded}
\begin{Highlighting}[]
\NormalTok{RF\_insu\_reg }\OtherTok{\textless{}{-}} \FunctionTok{randomForest}\NormalTok{(charges}\SpecialCharTok{\textasciitilde{}}\NormalTok{., }\AttributeTok{data=}\NormalTok{d\_train, }\AttributeTok{mtry=}\DecValTok{4}\NormalTok{, }\AttributeTok{importance=}\ConstantTok{TRUE}\NormalTok{)}
\FunctionTok{print}\NormalTok{(RF\_insu\_reg)}
\end{Highlighting}
\end{Shaded}

\begin{verbatim}
## 
## Call:
##  randomForest(formula = charges ~ ., data = d_train, mtry = 4,      importance = TRUE) 
##                Type of random forest: regression
##                      Number of trees: 500
## No. of variables tried at each split: 4
## 
##           Mean of squared residuals: 21319469
##                     % Var explained: 85.41
\end{verbatim}

\begin{Shaded}
\begin{Highlighting}[]
\FunctionTok{varImpPlot}\NormalTok{(RF\_insu\_reg)}
\end{Highlighting}
\end{Shaded}

\includegraphics{us_health_insurance_charges_project_files/figure-latex/question_3_random_forest-1.pdf}

\begin{Shaded}
\begin{Highlighting}[]
\NormalTok{RF\_cv\_MSE }\OtherTok{\textless{}{-}}\NormalTok{ RF\_insu\_reg}\SpecialCharTok{$}\NormalTok{mse[}\DecValTok{500}\NormalTok{]}
\FunctionTok{sprintf}\NormalTok{(}\StringTok{"Random forest MSE estimate based on OOB estimates from training data: \%s"}\NormalTok{,RF\_cv\_MSE)}
\end{Highlighting}
\end{Shaded}

\begin{verbatim}
## [1] "Random forest MSE estimate based on OOB estimates from training data: 21319469.2140666"
\end{verbatim}

\begin{Shaded}
\begin{Highlighting}[]
\NormalTok{predicted\_vals\_test }\OtherTok{\textless{}{-}} \FunctionTok{predict}\NormalTok{(RF\_insu\_reg, }\AttributeTok{newdata =}\NormalTok{ d\_test)}
\NormalTok{RF\_test\_MSE }\OtherTok{\textless{}{-}} \FunctionTok{mean}\NormalTok{((d\_test}\SpecialCharTok{$}\NormalTok{charges }\SpecialCharTok{{-}}\NormalTok{ predicted\_vals\_test)}\SpecialCharTok{\^{}}\DecValTok{2}\NormalTok{)}

\FunctionTok{sprintf}\NormalTok{(}\StringTok{"Tree MSE estimate based on testing data: \%s"}\NormalTok{, }\FunctionTok{round}\NormalTok{(RF\_test\_MSE,}\AttributeTok{digits=}\DecValTok{3}\NormalTok{))}
\end{Highlighting}
\end{Shaded}

\begin{verbatim}
## [1] "Tree MSE estimate based on testing data: 23890949.266"
\end{verbatim}

\hypertarget{boosting}{%
\subsubsection{Boosting}\label{boosting}}

\begin{Shaded}
\begin{Highlighting}[]
\NormalTok{boost\_insu\_reg }\OtherTok{\textless{}{-}} \FunctionTok{gbm}\NormalTok{(charges}\SpecialCharTok{\textasciitilde{}}\NormalTok{., }\AttributeTok{distribution=}\StringTok{"gaussian"}\NormalTok{, }\AttributeTok{data=}\NormalTok{d\_train, }\AttributeTok{n.trees=}\DecValTok{5000}\NormalTok{, }\AttributeTok{interaction.depth=}\DecValTok{4}\NormalTok{, }\AttributeTok{shrinkage =} \FloatTok{0.001}\NormalTok{)}

\CommentTok{\# Create a training control object for cross{-}validation}
\NormalTok{ctrl }\OtherTok{\textless{}{-}} \FunctionTok{trainControl}\NormalTok{(}\AttributeTok{method =} \StringTok{"cv"}\NormalTok{, }\AttributeTok{number =} \DecValTok{20}\NormalTok{)  }\CommentTok{\# 5{-}fold cross{-}validation}

\CommentTok{\# Use the train function from caret for cross{-}validation}
\NormalTok{cv\_results }\OtherTok{\textless{}{-}} \FunctionTok{train}\NormalTok{(charges }\SpecialCharTok{\textasciitilde{}}\NormalTok{ ., }\AttributeTok{data =}\NormalTok{ d\_train, }\AttributeTok{method =} \StringTok{"gbm"}\NormalTok{, }\AttributeTok{trControl =}\NormalTok{ ctrl, }\AttributeTok{verbose =} \ConstantTok{FALSE}\NormalTok{)}

\CommentTok{\# Access the cross{-}validated error values}
\NormalTok{boost\_cv\_MSE }\OtherTok{\textless{}{-}} \FunctionTok{mean}\NormalTok{(cv\_results}\SpecialCharTok{$}\NormalTok{results}\SpecialCharTok{$}\NormalTok{RMSE}\SpecialCharTok{\^{}}\DecValTok{2}\NormalTok{)}
\FunctionTok{sprintf}\NormalTok{(}\StringTok{"Boosted CV MSE estimate based on training data: \%s"}\NormalTok{,boost\_cv\_MSE)}
\end{Highlighting}
\end{Shaded}

\begin{verbatim}
## [1] "Boosted CV MSE estimate based on training data: 24635257.3156289"
\end{verbatim}

\begin{Shaded}
\begin{Highlighting}[]
\NormalTok{predicted\_vals\_test }\OtherTok{\textless{}{-}} \FunctionTok{predict}\NormalTok{(boost\_insu\_reg, }\AttributeTok{newdata =}\NormalTok{ d\_test)}
\end{Highlighting}
\end{Shaded}

\begin{verbatim}
## Using 5000 trees...
\end{verbatim}

\begin{Shaded}
\begin{Highlighting}[]
\NormalTok{boost\_test\_MSE }\OtherTok{\textless{}{-}} \FunctionTok{mean}\NormalTok{((d\_test}\SpecialCharTok{$}\NormalTok{charges }\SpecialCharTok{{-}}\NormalTok{ predicted\_vals\_test)}\SpecialCharTok{\^{}}\DecValTok{2}\NormalTok{)}
\FunctionTok{sprintf}\NormalTok{(}\StringTok{"Boosted MSE estimate based on testing data: \%s"}\NormalTok{,boost\_test\_MSE)}
\end{Highlighting}
\end{Shaded}

\begin{verbatim}
## [1] "Boosted MSE estimate based on testing data: 22739702.4477322"
\end{verbatim}

\begin{longtable}[]{@{}
  >{\raggedright\arraybackslash}p{(\columnwidth - 4\tabcolsep) * \real{0.2632}}
  >{\centering\arraybackslash}p{(\columnwidth - 4\tabcolsep) * \real{0.3684}}
  >{\centering\arraybackslash}p{(\columnwidth - 4\tabcolsep) * \real{0.3684}}@{}}
\toprule\noalign{}
\begin{minipage}[b]{\linewidth}\raggedright
Model
\end{minipage} & \begin{minipage}[b]{\linewidth}\centering
CV MSE
\end{minipage} & \begin{minipage}[b]{\linewidth}\centering
Test MSE
\end{minipage} \\
\midrule\noalign{}
\endhead
\bottomrule\noalign{}
\endlastfoot
Linear Model & \ensuremath{3.5678999\times 10^{7}} &
\ensuremath{4.0020943\times 10^{7}} \\
Lasso & \ensuremath{1.2810702\times 10^{8}} &
\ensuremath{1.3352178\times 10^{8}} \\
Trees & \ensuremath{1.6522123\times 10^{7}} &
\ensuremath{2.8823126\times 10^{7}} \\
Random Forest & \ensuremath{2.1319469\times 10^{7}} &
\ensuremath{2.3890949\times 10^{7}} \\
Boosting & \ensuremath{2.4635257\times 10^{7}} &
\ensuremath{2.2739702\times 10^{7}} \\
\end{longtable}

Based on the above metrics for test MSE, the best model to choose based
on sole prediction performance is the Boosted model because it gives the
lowest test MSE however the Random Forest has the lowest CV MSE
estimate.

If I was consulting for an insurance company, I would choose to use the
tree model because it's easier to interpret and the performance is not
that much worse than the boosted model. A tree is very dependent on the
training data sample and so we expect it to perform worse when compared
to other models such as the random forest or boosted model.

\end{document}
